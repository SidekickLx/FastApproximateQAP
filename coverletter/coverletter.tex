\documentclass{letter}
\signature{Joshua T. Vogelstein}
\address{3400 N. Charles St. \\ Whitehead Hall 105C \\ Baltimore, MD 21218 \\ USA}
\begin{document}
\begin{letter}{IEEE Transactions on Pattern Recognition and Machine Intelligence \\ Street\\ City\\ Country}
\opening{Dear Sir or Madam:}

We are delighted to submit to you two closely related manuscripts entitled, ``\emph{Shuffled Graph Classification: Theory and Connectome Applications}'' and ``\emph{Fast Inexact Graph Matching with Applications in Statistical Connectomics}''.  These two manuscripts are the final two-thirds of an initial trilogy of our work on graph classification.  This trilogy began with a manuscript currently under review at IEEE TPAMI entitled ``\emph{Graph Classification using Signal Subgraphs: Applications in Statistical Connectomics}''. 

The representation of data as graphs is gaining in popularity, as estimating connectedness between entities is becoming increasing popular.  This is true in fields as diverse of sociology, economics, as well as neuroscience, which is our motivating application.  Specifically, ``connectomics'' is an emerging field dedicated to studying whole organism neural circuitry.  While the data are burgeoning, rigorous statistical for the analysis of these data remains in its infancy.  We therefore are developing a suite of tools for such purposes that are (i) amenable to statistical theory and (ii) have improved performance on real world data.



This signal subgraph began started with the simplest scenario of interest, building a parametric classifier on graphs with labeled vertices.  To complement that work, we then attempted to build a non-parametric classifier on graphs with vertices lacking labels, which we describe in the ``Shuffled'' manuscript. As it turns out, universally consistent classifiers are available for such shuffled graphs, but they depend on solving a graph matching problem, which is known to be $\mathcal{NP}$-hard. Thus, we set out to develop a novel graph matching strategy that would scale sub $\mathcal{O}(n^3)$ for sparse graphs (such as those of interest).


This work was inspired by a very nice manuscript that had recently been published in IEEE TPAMI called ``A Path Following Algorithm for the Graph Matching Problem''.  That manuscript described a novel algorithm for graph matching with impressive performance on a standard set of benchmarks, with $\mathcal{O}(n^3)$ time complexity (besting the previous state of the art on 14 of 16 tests).   

Our interest in graph matching stems from our interest in graph classification, in particular, classifying large unlabeled graphs with vertices representing neurons in a mammalian brain (a field called ``connectomics'').  

\closing{With best regards,}
% \ps{P.S. Here goes your ps.}
% \encl{Enclosures.}
\end{letter}
\end{document}